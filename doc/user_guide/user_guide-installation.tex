\chapter{Installation \& Configuration}
\label{chap:installation}

Before you get started with OpTiMSoC you should notice that external
tools and libraries might be required that are in some cases
proprietary and cost some money. Although OpTiMSoC is developed at an
university with access to many EDA tools, we aim to always provide
tool flows and support for open and free tools, but especially when it
comes to synthesis such alternatives are even not available.

\section{Prerequisites}

Throughout this document some packages are required in your Linux
distribution. OpTiMSoC should principally work on all common Linux
distributions. In case you encounter problems in your system we highly
encourage you to contact the OpTiMSoC maintainers to fix these
problems. Nevertheless, we recommend Ubuntu 14.04 LTS as development
system and can ensure OpTiMSoC will work on it as we also work on
it. In the following we will refer to Ubuntu/Debian commands to
install packages, that will run under Ubuntu 14.04 LTS.

Independent of the OpTiMSoC components you plan to use, you will need some
packages to be installed:

\begin{lstlisting}
sudo apt-get -y install git python build-essential automake autoconf
\end{lstlisting}

\section{Option 1: Install the OpTiMSoC binary distribution (recommended)}

The most simple way to get started is with the release packages. You
can find the OpTiMSoC releases here:
\url{https://github.com/optimsoc/sources/releases}. With the release
you can find the distribution packets that can be extracted into any
directory and used directly from there. The recommended default is to
install OpTiMSoC into \verb|/opt/optimsoc|. There are two packages:
the \verb|base| package contains the programs, libraries and tools
to get started. The \verb|examples| package contains prebuilt verilator
simulations for the real quick start.

For example take the 2015.1 release and download both \verb|base| and
\verb|examples|. Then create the base folder and extract the package:

\begin{lstlisting}
sudo mkdir /opt/optimsoc
sudo chown $USER /opt/optimsoc
tar -xzf optimsoc-2015.1-base.tgz -C /opt/optimsoc
tar -xzf optimsoc-2015.1-examples.tgz -C /opt/optimsoc
\end{lstlisting}

You can now use the installation after setting the environment with
the pre-installed environment source script:

\begin{lstlisting}
cd /opt/optimsoc/2015.1
source optimsoc-environment.sh
\end{lstlisting}

We encourage you to put this code into your \verb|~/.bashrc|.

Now you are near to get started, but you need some programs to build
software to run in OpTiMSoC and execute the verilator-based
simulations. Those are: the \verb|or1k-elf-multicore| toolchain,
\verb|SystemC| and \verb|Verilator|. All those tools are free, but are
(except for an outdated \verb|Verilator| version) not part of the
Linux package systems. Hence you need to built those tools as
described in the Reference Manual, or you can simply download some
prebuilt-versions.

To do so simply run a script for your version:

\begin{lstlisting}
wget https://raw.githubusercontent.com/optimsoc/prebuilts/master/optimsoc-prebuilt-deploy.py
chmod a+x optimsoc-prebuilt-deploy.py
./optimsoc-prebuilt-deploy -d /opt/optimsoc systemc verilator or1kelf
\end{lstlisting}

You may of course leave out any of the tools if you already have it
installed. Finally, you will get another source script to set up the
environment for the prebuilt tools:

\begin{lstlisting}
source /opt/optimsoc/setup_prebuilt.sh
\end{lstlisting}

You are now ready to go to the tutorials.

\section{Option 2: Build OpTiMSoC from sources}

You can also build OpTiMSoC from the sources. This options usually
becomes standard if you start developing for or around OpTiMSoC. The
build is done from one git repository:
\url{https://github.com/optimsoc/sources}.

It is generally a good idea to understand git, especially when you
plan to contribute to OpTiMSoC. Nevertheless, we will give a more
detailed explanation of how to get your personal copies of OpTiMSoC
and (potentially) update them.

The start of a successful built is to install the tools
\verb|Verilator|, \verb|SystemC|, and the \verb|or1k-elf-multicore|
toolchain. The most simple way is to start with the prebuilt tools as
described above, then set the environment for the tools.

\begin{docnote}
During the installation, you'll frequently encounter three types of
directories.

\begin{itemize}
 \item \textbf{The source directory.} This is the place where the
   uncompiled source code files are stored. Usually, that is the folder that
   you cloned from the git repository.

   The \verb|$OPTIMSOC_SOURCE| environment variable should point to
   the root of the source directories.
 \item \textbf{The build or object directory.} For different
   components such a directory is used to build the component. The
   installer performs the build process in these directories and you
   perform individual builds there if you develop for OpTiMSoC (for
   example: if you develop the libraries).

   Sometimes, this directory is equal to the source directory, but
   most of the time, we create a new directory called \texttt{build}
   inside the source directory. Doing so has a great benefit: if
   something in the build process went wrong, you can simply delete
   the build directory and start all over again.

 \item \textbf{The installation directory.} This is the target
   directory where results of the build process are stored for further
   use. It is used by the installer or you when running \texttt{make
     install} to store the files generated by the build process. The
   environment variable \verb|$OPTIMSOC| points to the root of the
   installation directory.
\end{itemize}

\end{docnote}

\subsection{Prerequisites}

You will need some programs to build OpTiMSoC, e.g., on Ubuntu 14.04:

\begin{lstlisting}
sudo apt-get install autoconf automake libtool tcl texlive texlive-latex-extra texlive-fonts-extra
\end{lstlisting}

\subsection{Get the sources}

Start with checking out the repository:

\begin{lstlisting}
git clone https://github.com/optimsoc/sources.git optimsoc-sources
cd optimsoc-sources
\end{lstlisting}

\subsection{Build the code}

OpTiMSoC contains a Makefile which controls the whole build process. Building is
as simple as calling (inside the top-level source directory)

\begin{lstlisting}
make
\end{lstlisting}

\subsection{Install the code}
To install OpTiMSoC to the default location in \verb|/opt/optimsoc/VERSION|
it's sufficient to run

\begin{lstlisting}
sudo make install
\end{lstlisting}

You can also modify the target directory using environment variables passed to
make. This is especially useful if you don't have enough permissions to
write to \verb|/opt/optimsoc|.

\begin{itemize}
  \item Use \verb|INSTALL_PREFIX| to change the installation prefix from
    \verb|/opt/optimsoc| to something else. The installation will then go
    into \verb|INSTALL_PREFIX/VERSION|.
  \item Use \verb|INSTALL_TARGET| to fully override the installation path.
    The installation will then go exactly into this directory.
\end{itemize}

\begin{lstlisting}
# recommended option: use INSTALL_PREFIX
make install INSTALL_PREFIX=~/optimsoc

# full control for special cases: use INSTALL_TARGET
make install INSTALL_TARGET=~/optimsoc-testversion
\end{lstlisting}
